\section{Set Operations}

Set operations are ways that two sets can be combined into a new set. The
foundational set operations are union, intersection, and difference. Union
combines all the elements from both sets. Intersection results in only the
elements common to both sets. And difference takes the elements of one
set away from another set.

\begin{definition}{Union}{union}
  The \textbf{union} of two sets \( A \) and \( B \) is the set of all elements
  that are in either \( A \) or \( B \) or both. This is denoted by \( A \cup B \).
\end{definition}

Union examples:
\begin{itemize}
  \item \( A = \{ 1, 2, 3 \} \) and \( B = \{ 3, 4, 5 \} \), then \( A \cup B = \{ 1, 2, 3, 4, 5 \} \).
  \item \( A = \{ x, y, z \} \) and \( B = \{ a, b, c \} \), then \( A \cup B = \{ x, y, z, a, b, c \} \).
  \item \( A = \{ \text{Apple}, \text{Orange} \} \) and \( B = \{ \text{Banana}, \text{Grape} \} \), then
  \( A \cup B = \{ \text{Apple}, \text{Orange}, \text{Banana}, \text{Grape} \} \).
  \item \( A = \{ x \in \mathbb{R} \mid x < -10 \} \) and \( B = \{ x \in \mathbb{R} \mid x > 10 \} \), then
  \( A \cup B = \{ x \in \mathbb{R} \mid x < -10 \text{ or } x > 10 \} \).
\end{itemize}

\begin{definition}{Intersection}{intersection}
  The \textbf{intersection} of two sets \( A \) and \( B \) is the set of all
  elements that are in both \( A \) and \( B \). This is denoted by \( A \cap B \).
\end{definition}

Intersection examples:
\begin{itemize}
  \item \( A = \{ 1, 2, 3 \} \) and \( B = \{ 3, 4, 5 \} \), then \( A \cap B = \{ 3 \} \).
  \item \( A = \{ x, y, z \} \) and \( B = \{ a, b, c \} \), then \( A \cap B = \emptyset \).
  \item \( A = \{ \text{Apple}, \text{Orange} \} \) and \( B = \{ \text{Banana}, \text{Orange}, \text{Grape} \} \), then
  \( A \cap B = \{ \text{Orange} \} \).
  \item \( A = \{ x \in \mathbb{R} \mid x < 10 \} \) and \( B = \{ x \in \mathbb{R} \mid x > 5 \} \), then
  \( A \cap B = \{ x \in \mathbb{R} \mid 5 < x < 10 \} \).
\end{itemize}

\begin{definition}{Difference}{difference}
  The \textbf{difference} of two sets \( A \) and \( B \) is the set of all
  elements that are in \( A \) but not in \( B \). This is denoted by \( A - B \).
\end{definition}

The following table shows the definition of the set operations using set builder notation.
\begin{table}[H]
  \centering
  \begin{tabular}{p{1in} p{3in}}
    \toprule
    \textbf{Set Operation} & \textbf{Definition in Set Builder Notation} \\
    \midrule
    Union & \( A \cup B = \{ x \mid x \in A \text{ or } x \in B \} \) \\
    Intersection & \( A \cap B = \{ x \mid x \in A \text{ and } x \in B \} \) \\
    Difference & \( A - B = \{ x \mid x \in A \text{ and } x \notin B \} \) \\
    \bottomrule
  \end{tabular}
  \caption{Set Operations in Set Builder Notation}
\end{table}

\begin{definition}{Cartesian Product}{cartesianProduct}
  The \textbf{Cartesian product} of two sets \( A \) and \( B \) is the set of all
  ordered pairs \( (a, b) \) where \( a \in A \) and \( b \in B \). This is denoted by
  \( A \times B \).
  \medskip
  When a set is taken in a Cartesian product with itself, exponential notation is used:
  \begin{align*}
    A^2 &= A \times A \\
    A^3 &= A \times A \times A \\
    &\vdots \\
    A^n &= A \times A \times \ldots \times A
  \end{align*}
\end{definition}
Cartesian product examples:
\begin{itemize}
  \item \( A = \{ 1, 2 \} \) and \( B = \{ 3, 4 \} \), then
    \( A \times B = \{ (1, 3), (1, 4), (2, 3), (2, 4) \} \).
  \item \( C = \{ x, y \} \) and \( D = \{ a, b, c \} \),
    then \( C \times D = \{ (x, a), (x, b), (x, c), (y, a), (y, b), (y, c) \} \).
  \item \( E = \{ \text{Apple}, \text{Orange} \} \)
     and \( F = \{ 1, 2 \} \), then \\
     \( E \times F = \{ (\text{Apple}, 1), (\text{Apple}, 2), (\text{Orange}, 1), (\text{Orange}, 2) \} \).
  \item \( G = \{ 1, 2 \} \) and \( H = \{ x, y, z \} \), then \( G \times H = \{ (1, x), (1, y), (1, z), (2, x), (2, y), (2, z) \} \).
\end{itemize}

Looking at a common Cartesian product, \( \mathbb{R}^2 \), we can precisely describe it
as
\[
  \mathbb{R}^2 = \mathbb{R} \times \mathbb{R} = \{ (x, y) \mid x, y \in \mathbb{R} \}
\].
