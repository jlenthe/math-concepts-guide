\section{Set Inclusion}
Subsets and supersets describe the relationship between sets when one set is contained in
another set.

\begin{definition}{Subset}{subset}
  A set \( A \) is a \textbf{subset} of a set \( B \) if every element of
  \( A \) is also an element of \( B \).  This is denoted by \( A \subseteq B \).
\end{definition}

\begin{definition}
    {Superset}{superset}
    A set \( A \) is a \textbf{superset} of a set \( B \) if every element of
    \( B \) is also an element of \( A \). This is denoted by \( A \supseteq B \).
\end{definition}

Key notes about subsets and supersets:
\begin{itemize}
  \item Every set is a subset of itself, \( A \subseteq A \).
  \item Every set is a superset of itself, \( A \supseteq A \).
  \item The empty set is a subset of every set, \( \emptyset \subseteq A \).
  \item Every set is a superset of the empty set, \( A \supseteq \emptyset \).
\end{itemize}


\begin{definition}{Proper Subset}{properSubset}
  A set \( A \) is a \textbf{proper subset} of a set \( B \) if every element of
  \( A \) is also an element of \( B \) and \( A \neq B \). This is denoted by
  \( A \subset B \).
\end{definition}

\begin{definition}
    {Proper Superset}{properSuperset}
    A set \( A \) is a \textbf{proper superset} of a set \( B \) if every element of
    \( B \) is also an element of \( A \) and \( A \neq B \). This is denoted by
    \( A \supset B \).
\end{definition}

When we intuitively think of subsets and supersets, we think of one set being
fully contained in another set without being the same size as it. That intuition
corresponds to the concept of \emph{proper} subsets and \emph{proper} supersets,
since formally a set is a subset and superset of itself.

\begin{definition}{Power Set}{powerSet}
  The \textbf{power set} of a set \( A \) is the set of all possible subsets of \( A \).
  This is denoted by \( \mathcal{P}(A) \).
\end{definition}

Power set examples:
\begin{itemize}
  \item \( A = \{ 1, 2 \} \), then \( \mathcal{P}(A) = \{ \emptyset, \{ 1 \}, \{ 2 \}, \{ 1, 2 \} \} \).
  \item \( B = \{ x, y, z \} \), then \( \mathcal{P}(A) = \{ \emptyset, \{ x \}, \{ y \}, \{ z \},
  \{ x, y \}, \{ x, z \}, \{ y, z \}, \{ x, y, z \} \} \).
  \item \( C = \{ \text{Apple}, \text{Orange} \} \), then \( \mathcal{P}(A) = \{ \emptyset,
  \{ \text{Apple} \}, \{ \text{Orange} \}, \{ \text{Apple}, \text{Orange} \} \} \).
  \item \( D = \{ 1, 2, 3 \} \), then \( \mathcal{P}(A) = \{ \emptyset, \{ 1 \}, \{ 2 \},
  \{ 3 \}, \{ 1, 2 \}, \{ 1, 3 \}, \{ 2, 3 \}, \{ 1, 2, 3 \} \} \).
\end{itemize}

The cardinality of the power set of a set \( A \) is \( 2^{|A|} \).
