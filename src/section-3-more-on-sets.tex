\section{More on Sets}
Subsets and supersets describe the relationship between sets when one set is contained in
another set.

\begin{definition}{Subset}{subset}
  A set \( A \) is a \textbf{subset} of a set \( B \) if every element of
  \( A \) is also an element of \( B \).  This is denoted by \( A \subseteq B \).
\end{definition}

\begin{definition}
    {Superset}{superset}
    A set \( A \) is a \textbf{superset} of a set \( B \) if every element of
    \( B \) is also an element of \( A \). This is denoted by \( A \supseteq B \).
\end{definition}

Some key points to know about subsets and supersets include:
\begin{itemize}
  \item Every set is a subset of itself, \( A \subseteq A \).
  \item Every set is a superset of itself, \( A \supseteq A \).
  \item The empty set is a subset of every set, \( \emptyset \subseteq A \).
  \item Every set is a superset of the empty set, \( A \supseteq \emptyset \).
\end{itemize}


\begin{definition}{Proper Subset}{properSubset}
  A set \( A \) is a \textbf{proper subset} of a set \( B \) if every element of
  \( A \) is also an element of \( B \) and \( A \neq B \). This is denoted by
  \( A \subset B \).
\end{definition}

\begin{definition}
    {Proper Superset}{properSuperset}
    A set \( A \) is a \textbf{proper superset} of a set \( B \) if every element of
    \( B \) is also an element of \( A \) and \( A \neq B \). This is denoted by
    \( A \supset B \).
\end{definition}

When we intuitively think of subsets and supersets, we think of one set being
fully contained in another set without being the same size as it. That intuition
corresponds to the concept of \emph{proper} subsets and \emph{proper} supersets,
since formally a set is a subset and superset of itself.

\begin{definition}{Power Set}{powerSet}
  The \textbf{power set} of a set \( A \) is the set of all possible subsets of \( A \).
  This is denoted by \( \mathcal{P}(A) \).
\end{definition}

\begin{table}[H]
  \centering
  \begin{tabular}{p{1.5in} p{3.5in}}
    \toprule
    \textbf{Set} & \textbf{Power Set} \\
    \midrule
    \( A = \{ 1, 2 \} \) & \( \mathcal{P}(A) = \{ \emptyset, \{ 1 \}, \{ 2 \}, \{ 1, 2 \} \} \) \\
    \( B = \{ x, y, z \} \) & \( \mathcal{P}(B) = \{ \emptyset, \{ x \}, \{ y \}, \{ z \}, \{ x, y \}, \{ x, z \}, \{ y, z \}, \{ x, y, z \} \} \) \\
    \( C = \{ \text{Apple}, \text{Orange} \} \) & \( \mathcal{P}(C) = \{ \emptyset, \{ \text{Apple} \}, \{ \text{Orange} \}, \{ \text{Apple}, \text{Orange} \} \} \) \\
    \( D = \{ 1, 2, 3 \} \) & \( \mathcal{P}(D) = \{ \emptyset, \{ 1 \}, \{ 2 \}, \{ 3 \}, \{ 1, 2 \}, \{ 1, 3 \}, \{ 2, 3 \}, \{ 1, 2, 3 \} \} \) \\
    \bottomrule
  \end{tabular}
  \caption{Examples of Power Sets}
\end{table}

The cardinality of the power set of a set \( A \) is \( 2^{|A|} \) so the size of the
power set grows exponentially with the size of the original set.

There are several ways that two sets can be combined to form a new set. The
foundational set operations are union, intersection, and difference.

\begin{definition}{Union}{union}
  The \textbf{union} of two sets \( A \) and \( B \) is the set of all elements
  that are in either \( A \) or \( B \) or both. This is denoted by \( A \cup B \).
\end{definition}

The set union operation gives us a way to take two sets and combine them into a new
potentially larger set.

\begin{table}[H]
  \centering
  \begin{tabular}{ll}
    \toprule
    \textbf{Sets} & \textbf{Union} \\
    \midrule
    \( A = \{ 1, 2, 3 \}, B = \{ 3, 4, 5 \} \) & \( A \cup B = \{ 1, 2, 3, 4, 5 \} \) \\
    \( A = \{ x, y, z \}, B = \{ a, b, c \} \) & \( A \cup B = \{ x, y, z, a, b, c \} \) \\
    \( A = \{ \text{Apple}, \text{Orange} \}, B = \{ \text{Kiwi}, \text{Grape} \} \) & \( A \cup B = \{ \text{Apple}, \text{Orange}, \text{Kiwi}, \text{Grape} \} \) \\
    \( A = \{ x \in \mathbb{R} \mid x < -10 \},\ B = \{ x \in \mathbb{R} \mid x > 10 \} \) & \( A \cup B = \{ x \in \mathbb{R} \mid x < -10 \text{ or } x > 10 \} \) \\
    \bottomrule
  \end{tabular}
  \caption{Examples of Set Union}
\end{table}

\begin{definition}{Intersection}{intersection}
  The \textbf{intersection} of two sets \( A \) and \( B \) is the set of all
  elements that are in both \( A \) and \( B \). This is denoted by \( A \cap B \).
\end{definition}

The set intersection operation gives us a way to take two sets and identify the elements
that are common to both sets.

\begin{table}[H]
  \centering
  \begin{tabular}{ll}
    \toprule
    \textbf{Sets} & \textbf{Intersection} \\
    \midrule
    \( A = \{ 1, 2, 3 \},\ B = \{ 3, 4, 5 \} \) & \( A \cap B = \{ 3 \} \) \\
    \( A = \{ x, y, z \},\ B = \{ a, b, c \} \) & \( A \cap B = \emptyset \) \\
    \( A = \{ \text{Apple}, \text{Orange} \},\ B = \{ \text{Banana}, \text{Orange}, \text{Grape} \} \) & \( A \cap B = \{ \text{Orange} \} \) \\
    \( A = \{ x \in \mathbb{R} \mid x < 10 \},\ B = \{ x \in \mathbb{R} \mid x > 5 \} \) & \( A \cap B = \{ x \in \mathbb{R} \mid 5 < x < 10 \} \) \\
    \bottomrule
  \end{tabular}
  \caption{Examples of Set Intersection}
\end{table}

\begin{definition}{Difference}{difference}
  The \textbf{difference} of two sets \( A \) and \( B \) is the set of all
  elements that are in \( A \) but not in \( B \). This is denoted by \( A - B \).
\end{definition}

The set difference operation gives us a way to take the elements of one set away from another set
leaving only the elements that are unique to the first set.

The following table shows the definition of the set operations using set builder notation.
\begin{table}[H]
  \centering
  \begin{tabular}{p{1in} p{3in}}
    \toprule
    \textbf{Set Operation} & \textbf{Definition in Set Builder Notation} \\
    \midrule
    Union & \( A \cup B = \{ x \mid x \in A \text{ or } x \in B \} \) \\
    Intersection & \( A \cap B = \{ x \mid x \in A \text{ and } x \in B \} \) \\
    Difference & \( A - B = \{ x \mid x \in A \text{ and } x \notin B \} \) \\
    \bottomrule
  \end{tabular}
  \caption{Set Operations in Set Builder Notation}
\end{table}

\begin{definition}{Cartesian Product}{cartesianProduct}
  The \textbf{Cartesian product} of two sets \( A \) and \( B \) is the set of all
  ordered pairs \( (a, b) \) where \( a \in A \) and \( b \in B \). This is denoted by
  \( A \times B \).
  \medskip
  When a set is taken in a Cartesian product with itself, exponential notation is used:
  \begin{align*}
    A^2 &= A \times A \\
    A^3 &= A \times A \times A \\
    &\vdots \\
    A^n &= A \times A \times \ldots \times A
  \end{align*}
\end{definition}

The Cartesian product gives us a way to take two sets and pair up the elements from each set together.
It also gives us a simple way to describe a tuple where each element of the tuple is drawn from a
particular set.

\begin{table}[H]
  \centering
  \begin{tabular}{ll}
    \toprule
    \textbf{Sets} & \textbf{Cartesian Product} \\
    \midrule
    \( A = \{ 1, 2 \},\ B = \{ 3, 4 \} \) & \( A \times B = \{ (1, 3),\ (1, 4),\ (2, 3),\ (2, 4) \} \) \\
    \( C = \{ x, y \},\ D = \{ a, b, c \} \) & \( C \times D = \{ (x, a),\ (x, b),\ (x, c),\ (y, a),\ (y, b),\ (y, c) \} \) \\
    \( E = \{ \text{Apple},\ \text{Kiwi} \},\ F = \{ 1, 2 \} \) & \( E \times F = \{ (\text{Apple}, 1),\ (\text{Apple}, 2),\ (\text{Kiwi}, 1),\ (\text{Kiwi}, 2) \} \) \\
    \( G = \{ 1, 2 \},\ H = \{ x, y, z \} \) & \( G \times H = \{ (1, x),\ (1, y),\ (1, z),\ (2, x),\ (2, y),\ (2, z) \} \) \\
    \bottomrule
  \end{tabular}
  \caption{Examples of Cartesian Products}
\end{table}

Looking at a common Cartesian product, \( \mathbb{R}^2 \), we can precisely describe it
as
\[
  \mathbb{R}^2 = \mathbb{R} \times \mathbb{R} = \{ (x, y) \mid x, y \in \mathbb{R} \}
\]
\noindent It represents all ordered pairs of real numbers and is useful for identifying points
in a two-dimensional space. Similarly, \( \mathbb{R}^3 \) identifies all points in a
three-dimensional space.

