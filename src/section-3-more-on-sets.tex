\section{More on Sets}
Subsets and supersets describe the relationship between sets when one set is contained in
another set.

\begin{definition}{Subset}{subset}
  A set \( A \) is a \textbf{subset} of a set \( B \) if every element of
  \( A \) is also an element of \( B \).  This is denoted by \( A \subseteq B \).
\end{definition}

\begin{definition}
    {Superset}{superset}
    A set \( A \) is a \textbf{superset} of a set \( B \) if every element of
    \( B \) is also an element of \( A \). This is denoted by \( A \supseteq B \).
\end{definition}

Key notes about subsets and supersets:
\begin{itemize}
  \item Every set is a subset of itself, \( A \subseteq A \).
  \item Every set is a superset of itself, \( A \supseteq A \).
  \item The empty set is a subset of every set, \( \emptyset \subseteq A \).
  \item Every set is a superset of the empty set, \( A \supseteq \emptyset \).
\end{itemize}


\begin{definition}{Proper Subset}{properSubset}
  A set \( A \) is a \textbf{proper subset} of a set \( B \) if every element of
  \( A \) is also an element of \( B \) and \( A \neq B \). This is denoted by
  \( A \subset B \).
\end{definition}

\begin{definition}
    {Proper Superset}{properSuperset}
    A set \( A \) is a \textbf{proper superset} of a set \( B \) if every element of
    \( B \) is also an element of \( A \) and \( A \neq B \). This is denoted by
    \( A \supset B \).
\end{definition}

When we intuitively think of subsets and supersets, we think of one set being
fully contained in another set without being the same size as it. That intuition
corresponds to the concept of \emph{proper} subsets and \emph{proper} supersets,
since formally a set is a subset and superset of itself.

\begin{definition}{Power Set}{powerSet}
  The \textbf{power set} of a set \( A \) is the set of all possible subsets of \( A \).
  This is denoted by \( \mathcal{P}(A) \).
\end{definition}

Power set examples:
\begin{itemize}
  \item \( A = \{ 1, 2 \} \), then \( \mathcal{P}(A) = \{ \emptyset, \{ 1 \}, \{ 2 \}, \{ 1, 2 \} \} \).
  \item \( B = \{ x, y, z \} \), then \( \mathcal{P}(A) = \{ \emptyset, \{ x \}, \{ y \}, \{ z \},
  \{ x, y \}, \{ x, z \}, \{ y, z \}, \{ x, y, z \} \} \).
  \item \( C = \{ \text{Apple}, \text{Orange} \} \), then \( \mathcal{P}(A) = \{ \emptyset,
  \{ \text{Apple} \}, \{ \text{Orange} \}, \{ \text{Apple}, \text{Orange} \} \} \).
  \item \( D = \{ 1, 2, 3 \} \), then \( \mathcal{P}(A) = \{ \emptyset, \{ 1 \}, \{ 2 \},
  \{ 3 \}, \{ 1, 2 \}, \{ 1, 3 \}, \{ 2, 3 \}, \{ 1, 2, 3 \} \} \).
\end{itemize}

The cardinality of the power set of a set \( A \) is \( 2^{|A|} \).

There are several ways that two sets can be combined to form a new set. The
foundational set operations are union, intersection, and difference.

\begin{definition}{Union}{union}
  The \textbf{union} of two sets \( A \) and \( B \) is the set of all elements
  that are in either \( A \) or \( B \) or both. This is denoted by \( A \cup B \).
\end{definition}

The set union operation gives us a way to take two sets and combine them into a new
potentially larger set.

Union examples:
\begin{itemize}
  \item \( A = \{ 1, 2, 3 \} \) and \( B = \{ 3, 4, 5 \} \), then \( A \cup B = \{ 1, 2, 3, 4, 5 \} \).
  \item \( A = \{ x, y, z \} \) and \( B = \{ a, b, c \} \), then \( A \cup B = \{ x, y, z, a, b, c \} \).
  \item \( A = \{ \text{Apple}, \text{Orange} \} \) and \( B = \{ \text{Banana}, \text{Grape} \} \), then
  \( A \cup B = \{ \text{Apple}, \text{Orange}, \text{Banana}, \text{Grape} \} \).
  \item \( A = \{ x \in \mathbb{R} \mid x < -10 \} \) and \( B = \{ x \in \mathbb{R} \mid x > 10 \} \), then
  \( A \cup B = \{ x \in \mathbb{R} \mid x < -10 \text{ or } x > 10 \} \).
\end{itemize}

\begin{definition}{Intersection}{intersection}
  The \textbf{intersection} of two sets \( A \) and \( B \) is the set of all
  elements that are in both \( A \) and \( B \). This is denoted by \( A \cap B \).
\end{definition}

The set intersection operation gives us a way to take two sets and identify the elements
that are common to both sets.

Intersection examples:
\begin{itemize}
  \item \( A = \{ 1, 2, 3 \} \) and \( B = \{ 3, 4, 5 \} \), then \( A \cap B = \{ 3 \} \).
  \item \( A = \{ x, y, z \} \) and \( B = \{ a, b, c \} \), then \( A \cap B = \emptyset \).
  \item \( A = \{ \text{Apple}, \text{Orange} \} \) and \( B = \{ \text{Banana}, \text{Orange}, \text{Grape} \} \), then
  \( A \cap B = \{ \text{Orange} \} \).
  \item \( A = \{ x \in \mathbb{R} \mid x < 10 \} \) and \( B = \{ x \in \mathbb{R} \mid x > 5 \} \), then
  \( A \cap B = \{ x \in \mathbb{R} \mid 5 < x < 10 \} \).
\end{itemize}

\begin{definition}{Difference}{difference}
  The \textbf{difference} of two sets \( A \) and \( B \) is the set of all
  elements that are in \( A \) but not in \( B \). This is denoted by \( A - B \).
\end{definition}

The set differece operation gives us a way to take the elements of one set away from another set
leaving only the elements that are unique to the first set.

The following table shows the definition of the set operations using set builder notation.
\begin{table}[H]
  \centering
  \begin{tabular}{p{1in} p{3in}}
    \toprule
    \textbf{Set Operation} & \textbf{Definition in Set Builder Notation} \\
    \midrule
    Union & \( A \cup B = \{ x \mid x \in A \text{ or } x \in B \} \) \\
    Intersection & \( A \cap B = \{ x \mid x \in A \text{ and } x \in B \} \) \\
    Difference & \( A - B = \{ x \mid x \in A \text{ and } x \notin B \} \) \\
    \bottomrule
  \end{tabular}
  \caption{Set Operations in Set Builder Notation}
\end{table}

\begin{definition}{Cartesian Product}{cartesianProduct}
  The \textbf{Cartesian product} of two sets \( A \) and \( B \) is the set of all
  ordered pairs \( (a, b) \) where \( a \in A \) and \( b \in B \). This is denoted by
  \( A \times B \).
  \medskip
  When a set is taken in a Cartesian product with itself, exponential notation is used:
  \begin{align*}
    A^2 &= A \times A \\
    A^3 &= A \times A \times A \\
    &\vdots \\
    A^n &= A \times A \times \ldots \times A
  \end{align*}
\end{definition}
Cartesian product examples:
\begin{itemize}
  \item \( A = \{ 1, 2 \} \) and \( B = \{ 3, 4 \} \), then
    \( A \times B = \{ (1, 3), (1, 4), (2, 3), (2, 4) \} \).
  \item \( C = \{ x, y \} \) and \( D = \{ a, b, c \} \),
    then \( C \times D = \{ (x, a), (x, b), (x, c), (y, a), (y, b), (y, c) \} \).
  \item \( E = \{ \text{Apple}, \text{Orange} \} \)
     and \( F = \{ 1, 2 \} \), then \\
     \( E \times F = \{ (\text{Apple}, 1), (\text{Apple}, 2), (\text{Orange}, 1), (\text{Orange}, 2) \} \).
  \item \( G = \{ 1, 2 \} \) and \( H = \{ x, y, z \} \), then \( G \times H = \{ (1, x), (1, y), (1, z), (2, x), (2, y), (2, z) \} \).
\end{itemize}

Looking at a common Cartesian product, \( \mathbb{R}^2 \), we can precisely describe it
as
\[
  \mathbb{R}^2 = \mathbb{R} \times \mathbb{R} = \{ (x, y) \mid x, y \in \mathbb{R} \}
\].

