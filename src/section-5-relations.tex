\section{Relations}
Relations are used to describe how elements of various sets
are related to each other. For example, comparisons of numbers such as
\( a < b \) can be defined as relations.

\begin{definition}{Relation}{relation}
  A \textbf{relation} on sets \( A \) and \( B \) is any subset of the
  Cartesian product \( A \times B \).
\end{definition}

Relation examples:
\begin{itemize}
  \item The less-than relation on \( \mathbb{R} \) may be defined as:
    \[
      \{ (x, y) \in \mathbb{R} \times \mathbb{R} \mid x < y \}
    \]
  \item The integer-squared relation on \( \mathbb{Z} \) may be defined as:
    \[
      \{ (x, y) \in \mathbb{Z} \times \mathbb{Z} \mid y = x^2 \}
    \]
  \item The natural square-root relation is a relation from \( \mathbb{N} \) to \( \mathbb{R} \)
  (i.e. a subset of \( \mathbb{N} \times \mathbb{R} \)) and may be defined as:
    \[
      \{ (x, y) \in \mathbb{N} \times \mathbb{R} \mid y = +\sqrt{x} \}
    \]
\end{itemize}

Relations often have special properties that are useful to identify.

\begin{definition}{Reflexive}{reflexive}
  A relation \( R \) on set \( A \) is \textbf{reflexive} if for all \( a \in A \), \( (a, a) \in R \).
\end{definition}

\begin{definition}{Irreflexive}{irreflexive}
  A relation \( R \) on set \( A \) is \textbf{irreflexive} if for all \( a \in A \), \( (a, a) \notin R \).
\end{definition}

A reflexive relation is one where every element is related to itself. An irreflexive relation is one where
no element is related to itself. A relation that is neither reflexive nor irreflexive has some elements
that are related to themselves and some that are not.

\begin{definition}{Symmetric}{symmetric}
  A relation \( R \) on set \( A \) is \textbf{symmetric} if for all \( a, b \in A \), if \( (a, b) \in R \) then
  \( (b, a) \in R \).
\end{definition}

A symmetric relation is one where the relation has no sense of direction. It indicates that two elements are
intrinsically mutually related to each other in some way.

\begin{definition}{Transitive}{transitive}
  A relation \( R \) on set \( A \) is \textbf{transitive} if for all \( a, b, c \in A \), if \( (a, b) \in R \) and
  \( (b, c) \in R \) then \( (a, c) \in R \).
\end{definition}

A transitive relation implies some kind of overarching structure or relatability among all the elements
such as the ability to strictly order numbers in a consistent way.

\begin{definition}{Equivalence Relation}{equivalence-relation}
  A relation \( R \) is an \textbf{equivalence relation} if it is reflexive, symmetric, and transitive.
\end{definition}

An equivalence relation, as defined above, specifies the properties a relation must have to embody
the general concept of "equivalence", but it does not specify the full meaning of a particular
equivalence. Multiple equivalence relations can be defined on the same set with different meanings.
For example, on the set of natural numbers \( \mathbb{N} \), we can define equivalence relations based
on numeric quantity, parity (even or odd), or divisibility. All such relations would have the properties
of reflexivity, symmetry, and transitivity.

For example, we could define an equivalence relation on \( \mathbb{N} \) that corresponds
to normal numerical equality, but we could also define parity equality as an equivalence relation where two
natural numbers are "equivalent" if they are both even or both odd.

\begin{table}[H]
  \centering
  \begin{tabular}{p{2in} p{3in}}
  \toprule
  \textbf{Comparison Relation} & \textbf{Properties} \\
  \midrule
  \( < \) (less than) & transitive, irreflexive \\
  \( \leq \) (less than or equal to) & reflexive, transitive \\
  \( > \) (greater than) & transitive, irreflexive \\
  \( \geq \) (greater than or equal to) & reflexive, transitive \\
  \( = \) (equal) & reflexive, symmetric, transitive \\
  \( \neq \) (not equal) & irreflexive, symmetric \\
  \bottomrule
  \end{tabular}
  \caption{Properties of Number Comparisons}
\end{table}

