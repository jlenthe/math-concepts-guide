\section{Sets}

\begin{definition}{Set}{set}
  A \textbf{set} is a collection of distinct objects.
\end{definition}

Finite sets may have all their elements enumerated in curly braces, such as:
\begin{align*}
  A &= \{ 1, 2, 3 \} \\
  B &= \{ x, y, z \} \\
  C &= \{ \text{Apple}, \text{Orange} \}
\end{align*}

Some key points about sets:
\begin{itemize}
  \item Sets may not have duplicate elements. This is what is meant by
  \emph{distinct objects}.
  \item Sets may contain objects of any type including numbers, strings, or other sets
   or contain objects of different types.
  \item Sets may have any number of elements including zero elements or be infinite.
\end{itemize}

\begin{definition}{Empty Set}{emptyset}
  The \textbf{empty set} or \textbf{null set} is the set that contains no elements
  and is denoted by \( \emptyset \) or \( \{ \} \).
\end{definition}

\begin{definition}{Cardinality}{cardinality}
  The \textbf{cardinality} of a finite set is the number of elements in the set and is
  denoted by \( |A| \) where \( A \) is the set. The cardinality of the empty
  set is zero, \( |\emptyset| = 0 \). The cardinality of an infinite set is simply said to be
  infinite.
\end{definition}

Cardinality examples:
\begin{itemize}
  \item \( A = \{ 1, 2, 3 \} \), then \( |A| = 3 \).
  \item \( B = \{ x, y, z \} \), then \( |B| = 3 \).
  \item \( C = \{ \text{Apple}, \text{Orange} \} \), then \( |C| = 2 \).
  \item \( D = \{ 1, 2, 3, 4, 5, 6, 7, 8, 9, 10 \} \), then \( |D| = 10 \).
  \item \( E \) is the set of all even integers, then \( |E| \) is infinite.
\end{itemize}

\begin{advancedTopic}
  Infinite sets do not all have the same cardinality. Infinite sets have different cardinalities
  that can be distinguished. For simplicity here, though, we just refer to infinite sets. For more
  information, look up the topic of \emph{transfinite cardinals}.
\end{advancedTopic}


Sets of numbers, such as the real numbers and integers, are commonly referred to and have
their own symbols as shown in the follow table.
\begin{table}[H]
  \centering
  \begin{tabular}{p{1in} p{2in}}
  \toprule
  \textbf{Set Name} & \textbf{Description} \\
  \midrule
  \(\mathbb{N}\) & Natural numbers \\
  \(\mathbb{Z}\) & Integers \\
  \(\mathbb{Q}\) & Rational numbers \\
  \(\mathbb{R}\) & Real numbers \\
  \(\mathbb{C}\) & Complex numbers \\
  \bottomrule
  \end{tabular}
  \caption{Symbols for Common Sets of Numbers}
\end{table}

\begin{definition}{Membership}{membership}
  The symbol \( \in \) is used to assert that an element is a member of a set.
  The symbol \( \notin \) is used to assert that an element is not a member of a set.
\end{definition}

For example, if \( A = \{ 1, 2, 3 \} \) then \( 1 \in A \) and \( 4 \notin A \) are both
true assertions. We may also say and \( 4 \in A \) is a false statement.

\begin{definition}{Set Builder Notation}{set-builder-notation}
  Sets expressed or defined by \textbf{set builder notation} are sets that are defined by
  the properties of their elements with the following form:
  \[
    A = \{ x \in S \mid P(x) \}
  \]
  where \( S \) is a set and \( P(x) \) is a property of the elements of \( S \).
\end{definition}

Sets may be defined by the properties of their elements and may be infinite. For example,
\[
  D = \{ x \in \mathbb{R} \mid x > 10 \}
\]
means that \( D \) is the set of all real numbers \( x \) such that \( x \) is greater than 10.
\begin{align*}
  E &= \{ n \in \mathbb{Z} \mid n \text{ is even} \} \\
  F &= \{ x \in \mathbb{N} \mid x^2 < 10 \} \\
  G &= \{ x \in \mathbb{R} \mid x^2 - 25 = 0 \}
\end{align*}
\( E \) is the set of all even integers and is an infinite set. \( F \) is the set all natural
numbers who square is less than 10 (\( F \) could also be expressed as \( F = \{ 1, 2, 3 \} \)).
\( G \) is the set of all real numbers, \( x \), that satisfy the equation \( x^2 - 25 = 0 \).
\( G \) happens to be a finite set with two elements, \( G = \{ 5, -5 \} \).

\begin{definition}{Set Equality}{setEquality}
  Two sets \( A \) and \( B \) are equal, denoted by \( A = B \), if and only if
  every element of \( A \) is also an element of \( B \) and every element of \( B \)
  is also an element of \( A \). Sets that are not equal are denoted by \( A \neq B \).
\end{definition}