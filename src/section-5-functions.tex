\section{Functions}

Functions are a construct that maps elements from one set to another.

\begin{definition}{Function}{function}
  A \textbf{function} \( f \) from a set \( A \) to a set \( B \) is a relation
  \( f \subseteq A \times B \) such that for every element \( a \) in \( A \),
  there is exactly one element \( b \) in \( B \) with \( (a, b) \in f \).
  The set \( A \) is called the \textbf{domain} of the function, and the set
  \( B \) is called the \textbf{codomain}. The value \( f(a) = b \) is called the
  \textbf{image} of \( a \), and is often read as ``\( f \) of \( a \)''.

  \medskip

  The set of all images of elements in \( A \) is called the \textbf{range} of
  the function.

  \medskip

  The domain and codomain of a function \( f \) may be denoted by \( f: A \to B \).
\end{definition}

Functions have many applications. Some examples of possible functions include:
\begin{itemize}
  \item Given a date, determine the day of the week.
  \item Given the radius of a circle, calculate its area.
  \item Given the price an item for sale, calculate its sales tax.
  \item Given a location on Earth (as latitude and longitude pair), determine its distance to
    New York City.
\end{itemize}

Some key notes about functions:
\begin{itemize}
  \item The range of a function is the set of all actual outputs and, by
    definition, is a subset of the codomain.
  \item It is common for the domain and codomain to be the same set, but it is not
    required.
  \item Specific functions may be defined by a formula, a set of properties that hold true,
    or an algorithm.
\end{itemize}

Often functions are intuitively thought as a process mapping an \textbf{input} to
an \textbf{output}. Using this terminology:
\begin{itemize}
  \item The input is an element of the domain.
  \item The output is an element of the codomain.
  \item The function is the process or method that maps the input to the output.
  \item The image is the output of the function.
  \item The range is the set of all possible outputs.
\end{itemize}

\begin{advancedTopic}
  The domain and codomain of a function can be any sets, including
  sets of functions themselves. A function that takes one or more functions
  as input and returns a function as output is called a \textbf{higher-order
  function}. These functions play a central role in computer science
  (e.g. functional programming) as well as in mathematics, where they
  appear in areas such as functional analysis and the calculus of variations.
\end{advancedTopic}

An example of a function defined by a formula is the function \( f: \mathbb{R} \to \mathbb{R} \)
defined by:
\[
  f(x) = x^2
\]
\noindent This function takes a real number \( x \) as input and returns the square of \( x \) as
output.

An example of a function defined by its properties is the absolute value function
\( f: \mathbb{R} \to \mathbb{R} \) defined as:
\[
  f(x) = \begin{cases}
    x & \text{if } x \geq 0 \\
    -x & \text{if } x < 0
  \end{cases}
\]

An example of a tabulated function is the function \( f: \mathbb{A} \to \mathbb{A} \)
where \( A = \{ 0, 1, 2, 3 \}\) defined by:
\begin{table}[H]
  \centering
  \begin{tabular}{cc}
    \toprule
    \textbf{\(n\)} & \textbf{\(f(n)\)} \\
    \midrule
    0 & 1 \\
    1 & 2 \\
    2 & 3 \\
    3 & 0 \\
    \bottomrule
  \end{tabular}
  \caption{Tabulated Function Example}
\end{table}
\noindent This function adds 1 to the input and wraps around to 0 when the input is 3.

Functions may also take tuples as input. Typically the domain set is written as a Cartesian
product. For example, the function \( f: \mathbb{R}^2 \to \mathbb{R} \) defined by:
\[
  f(x, y) = x^2 + y^2
\]
\noindent This function takes a pair of real numbers \( (x, y) \) as input and returns the sum of
the squares of \( x \) and \( y \) as output. Similarly, with 3 inputs \( f: \mathbb{R}^3 \to
\mathbb{R} \) defined by:
\[
  f(x, y, z) = x^2 + y^2 + z^2
\]

