\section{Numbers}
\label{sec:numbers}

\begin{definition}{Natural Numbers}{naturalNumbers}
  The \textbf{Natural Numbers} are a set of numbers used for counting a discrete
  quantity of objects. They are denoted by the symbol \(\mathbb{N}\).
  \[
    \mathbb{N} = \{0, 1, 2, 3, \ldots\}
  \]
\end{definition}

The Natural Numbers may also be expressed in recursive set builder notation as
\[
  \mathbb{N} = \{ 0 \} \cup \{ n + 1 \mid n \in \mathbb{N} \}
\]
In other words, we begin with 0 as a natural number, and then each subsequent number
is formed by adding 1 to a number already in the set. This reflects the idea of
natural numbers as the basic instrument of counting.

Note that some definitions of the natural numbers exclude zero, defining
\(\mathbb{N} = \{1, 2, 3, \ldots\}\). Here, we include zero as the smallest natural
number because we find it useful and intuitive to be able to quantify emptiness
or nothingness with a natural number.

\begin{definition}{Integers}{integers}
  The \textbf{Integers} are a set of numbers that include all the natural numbers
  and, in addition, their negatives.
  They are denoted by the symbol \(\mathbb{Z}\).
  \[
    \mathbb{Z} = \{\ldots, -3, -2, -1, 0, 1, 2, 3, \ldots\}
  \]
\end{definition}

The Integers may also be expressed in set builder notation by building on the Natural Numbers
and adding their negatives to the set:
\[
  \mathbb{Z} = \mathbb{N} \cup \{-n \mid n \in \mathbb{N}, n > 0\}
\]

\begin{definition}{Real Numbers}{realNumbers}
  The \textbf{Real Numbers}, denoted by \(\mathbb{R}\), are the set of values that can be placed
  on an infinite, continuous, ordered number line. They include all quantities that represent
  continuous magnitudes such as length, time, and mass—in contrast to discrete quantities like
  counts or whole numbers.
\end{definition}

This definition may seem a bit vague, so let's elaborate on it. An ordered number line is a
one-dimensional space that extends infinitely in both directions and any number can be
ordered relative to any other number. For example, if \( a \) and \( b \) are two real numbers,
then either \(a < b\), \(a = b\), or \(a > b\). A continuous number line means that there are
no gaps between the numbers. For example, between any two real numbers \( a \) and \( b \), there
exists another real number \( c \) such that \(a < c < b\).

\begin{definition}{Rational Numbers}{rationalNumbers}
  The \textbf{Rational Numbers} are the subset of real numbers that can be expressed as the
  ratio of two integers, where the denominator is not zero. They are denoted by
  the symbol \(\mathbb{Q}\).
  \[
    \mathbb{Q} = \left\{ \frac{a}{b} \,\middle|\, a, b \in \mathbb{Z},\ b \ne 0 \right\}
  \]

  The term "rational" here means "ratio" and does not refer to "reason" or "rationality".
\end{definition}

Rational numbers are the ordinary decimals and fractions that we encounter in everyday life, and include
natural numbers and integers. Some examples include:
\begin{itemize}
\item \(\frac{1}{2}\)
\item \(0.5\)
\item \(-3\)
\item \(\frac{3}{1}\)
\item \(2.75\)
\item \(\frac{11}{4}\)
\item \(0.3333\ldots\)
\end{itemize}

Infinitely repeating decimals, such as \(0.3333\ldots\) and \(0.\overline{12345}\), are also rational
numbers since they can be expressed as the integer ratios \(\frac{1}{3}\) and \(\frac{12345}{99999}\),
respectively. If the decimal representation of a number never terminates or repeats, then it is
not a rational number.

\begin{advancedTopic}
  People are often surprised to learn that \( 0.999\ldots \) is equal to \( 1 \). The proof of this
  fact is easily understood and well worth reading.
\end{advancedTopic}

\begin{definition}{Irrational Numbers}{irrationalNumbers}
  The \textbf{Irrational Numbers} are the subset of numbers that cannot be expressed
  as the ratio of two integers. They are denoted by the symbol \(\mathbb{R} \setminus \mathbb{Q}\).
  \[
    \mathbb{R} \setminus \mathbb{Q} = \left\{ x \,\middle|\, x \in \mathbb{R},\ x \notin \mathbb{Q} \right\}
  \]
\end{definition}

Irrational numbers include numbers like \(\sqrt{2}\), \(\pi\), and \( e \), which have been proven to
have no representation as a ratio of integers. Irrational numbers may be defined conceptually. For
example, \(\pi\) is defined as the ratio of the circumference of a circle to its diameter. Irrational
numbers are often handled abstractly through concepts and logic, or approximated numerically by rational numbers.

\begin{advancedTopic}
  The existence and nature of irrational numbers opens up a wide range of questions to explore. How can we know
  whether a number defined conceptually is rational or irrational? How can we approximate a given irrational
  number? Are there any irrational number that cannot be approximated? Which are there more of—rational or
  irrational numbers?
\end{advancedTopic}

We can express the set inclusion of all the basic number sets as follows:
\[
  \mathbb{N} \subset \mathbb{Z} \subset \mathbb{Q} \subset \mathbb{R}
\]
