\section{Logic}

Mathematical logic gives us a way to express statements involving
strictly true and false values.

\begin{definition}{Propositional Variable}{propositionalVariable}
  A \textbf{propositional variable} is a variable with a value from the set
  \( \mathbb{B} = \{ \text{True}, \text{False} \} \). A propositional variable
  is sometimes called a Boolean variable.
\end{definition}

\begin{definition}{Proposition}{proposition}
  A \textbf{proposition} is a statement that is either true or false but not both.
\end{definition}

A proposition is so called because it may be a question to be answered,
a statement of truth or falsehood, or a claim to be proved or disproved
depending on the context.

Propositional variables may be combined to form more complex propositions with
logical operators. The most common and familiar logical operators are \textbf{and}, \textbf{or},
and \textbf{not}. These operators have special symbols as delineated in the
following table:
\begin{table}[H]
  \centering
  \begin{tabular}{cc}
    \toprule
    Logical Operator & Symbol \\
    \midrule
    and & \( \land \) \\
    or & \( \lor \) \\
    not & \( \lnot \) \\
    \bottomrule
  \end{tabular}
  \caption{Logical Operators and Symbols}
\end{table}

\begin{definition}{Logical Operator}{logicalOperator}
  A \textbf{logical operator} is a function \( f: \mathbb{B}^n \to \mathbb{B} \),
  for \( n \in \mathbb{N} \) and \( n \ge 1 \) that takes one or more propositional variables
  as input and returns a value in \( \mathbb{B} \).
\end{definition}

Usually logical operators take only 1 input (unary) or 2 inputs (binary). For such operators,
there are only 2 or 4 possible inputs, respectively, so typically such operators are
tabulated in a \emph{truth table}. The following truth tables define the "and", "or"
and "not" logical operators, respectively.
\begin{table}[H]
  \centering
  \begin{tabular}{cc}
    \toprule
   \(P\) & \( \lnot P \)  \\
    \midrule
    True & False \\
    False & True \\
    \bottomrule
  \end{tabular}
  \caption{Truth Table for "not"}
\end{table}

\begin{table}[H]
  \centering
  \begin{tabular}{ccc}
    \toprule
    \(P\) & \(Q\) & \( P \land Q \)  \\
    \midrule
    True & True & True \\
    True & False & False \\
    False & True & False \\
    False & False & False \\
    \bottomrule
  \end{tabular}
  \caption{Truth Table for "and"}
\end{table}

\begin{table}[H]
  \centering
  \begin{tabular}{ccc}
    \toprule
    \(P\) & \(Q\) & \( P \lor Q \)  \\
    \midrule
    True & True & True \\
    True & False & True \\
    False & True & True \\
    False & False & False \\
    \bottomrule
  \end{tabular}
  \caption{Truth Table for "or"}
\end{table}

\begin{advancedTopic}
  While "and" and "or" are by far the most common logical binary operators,
  there are 16 binary logical operators in total. Some of these are important
  in digital circuit design and construction. For more information, look up
  the NAND, NOR, XOR, and XNOR operators or digital logic gates.
\end{advancedTopic}

Another logical operator of utmost importance is the \textbf{implication} operator,
which is denoted by the symbol \( \implies \). The proposition \( P \implies Q \),
means that if \( P \) is true, \( Q \) must also be true. We can define this more precisely
with a truth table:
\begin{table}[H]
  \centering
  \begin{tabular}{ccc}
    \toprule
    \(P\) & \(Q\) & \( P \implies Q \)  \\
    \midrule
    True & True & True \\
    True & False & False \\
    False & True & True \\
    False & False & True \\
    \bottomrule
  \end{tabular}
  \caption{Truth Table for "implication"}
\end{table}

\begin{advancedTopic}
  The notion that "\( \text{False} \implies \text{True} \) is True" is quite
  counterintuitive. It seems at odds with the typical meaning of the word "implies" in
  natural language and philosophical argumentation. This has been the subject of
  much philosophical discussion. For more information, look up \emph{material
  implication paradox}.
\end{advancedTopic}



