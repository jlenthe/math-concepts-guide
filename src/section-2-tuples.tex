\section{Tuples}
\begin{definition}{Tuple}{tuple}
  A \textbf{tuple} is an finite ordered list of elements.
\end{definition}

Tuples are denoted by parentheses, such as:
\begin{align*}
  A &= (1, 2, 3) \\
  B &= (x, y, z) \\
  C &= (\text{Apple}, \text{Orange}) \\
  D &= (1, 2, 3, 4, 5, 6, 7, 8, 9, 10)
\end{align*}

Some key points about tuples:
\begin{itemize}
  \item A tuples may have repeated elements. There is no requirement that the elements be distinct.
  \item A tuple may have zero elements.
  \item The number of elements in a tuple is called the \textbf{length} of the tuple.
  \item A tuple may contain objects of any type including numbers, strings, sets, or other tuples.
  \item Tuples of certain lengths have special names (see below).
\end{itemize}

\begin{table}[H]
  \centering
  \begin{tabular}{ll}
  \toprule
  \textbf{Tuple Length} & \textbf{Names} \\
  \midrule
  0 & empty tuple \\
  1 & singleton, single \\
  2 & ordered pair \\
  3 & triple, triplet, triad \\
  4 & quadruple \\
  5 & quintuple \\
  6 & sextuple \\ 
  7 & septuple \\
  8 & octuple \\
  \( n  \) & n-tuple \\
  \bottomrule
  \end{tabular}
  \caption{Names for Common Tuples}
\end{table}

Tuple equality is defined as follows:
\begin{definition}{Tuple Equality}{tupleEquality}
  Two tuples \( A \) and \( B \) are equal if and only if they have the same length and 
  their corresponding elements are equal.
\end{definition}
Tuple equality examples:
\begin{itemize}
  \item \( A = (1, 2, 3) \) and \( B = (1, 2, 3) \), then \( A = B \).
  \item \( A = (1, 2, 3) \) and \( B = (3, 2, 1) \), then \( A \neq B \).
  \item \( A = (x, y, z) \) and \( B = (x, y, z) \), then \( A = B \).
  \item \( A = (x, y, z) \) and \( B = (x, z, y) \), then \( A \neq B \).
  \item \( A = (\text{Apple}, \text{Orange}) \) and \( B = (\text{Apple}, \text{Orange}) \), then \( A = B \).
  \item \( A = (\text{Apple}, \text{Orange}) \) and \( B = (\text{Orange}, \text{Apple}) \), then \( A \neq B \).
  \item \( A = (1, 2, 3) \) and \( B = (1, 2) \), then \( A \neq B \).
\end{itemize}

A tuple may be specified by a sequence of elements with a subscripted index and an ellipsis. 
For example, a tuple with n
\(n\) elements (\( n \ge 1\)) may be specified as:
\[
  A = (a_1, \ldots, a_n) 
\]