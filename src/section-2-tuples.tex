\section{Tuples}
\begin{definition}{Tuple}{tuple}
  A \textbf{tuple} is a finite ordered list of elements.
\end{definition}

The word "tuple" is commonly pronounced as either “tuh-ple” or “too-ple”.

Tuples have many applications. Tuples can represent the coordinates of a point in space,
the state of a system, or the configuration of a machine. Tuples are also the building
blocks of other basic mathematical concepts such as relations and functions.

A tuple is often denoted using indexed elements, such as:
\[
  A = (a_1, \ldots, a_n)
\]
where each \( a_i \) is an element of the tuple, indexed by its position.

\begin{table}[H]
  \centering
  \begin{tabular}{ll}
    \toprule
    \textbf{Tuple} & \textbf{Description} \\
    \midrule
    \( A = (1, 2, 3) \) & Tuple with elements \( A_1 = 1 \), \( A_2 = 2 \), and \( A_3 =3 \) \\
    \( B = (x, y, z) \) & Tuple with elements \( B_1 = x \), \( B_2 = y \), and \( B_3 = z \) \\
    \( C = (\text{Apple}, \text{Orange}) \) & Tuple with elements \( C_1 = \text{Apple} \), \( C_2 = \text{Orange} \) \\
    \( D = (7, \{1, 2\}, (a, b)) \) & Tuple with elements \( D_1 = 7 \), \( D_2 = \{1, 2\} \), and \( D_3 = (a, b) \) \\
    \bottomrule
  \end{tabular}
  \caption{Tuple Examples}
\end{table}

Some key points to know about tuples include:
\begin{itemize}
  \item A tuple may have repeated elements. There is no requirement that the elements be distinct.
  \item A tuple may have zero elements.
  \item The number of elements in a tuple is called the \textbf{length} of the tuple.
  \item A tuple may contain objects of any type including numbers, strings, sets, or other tuples
   and elements of different types are allowed in the same tuple.
\end{itemize}

Tuples of specific lengths may have special names. The following table lists the names of tuples
of common lengths. Aside from \emph{ordered pair}, the names are not universally accepted and may
vary by author or context. So it is often best to simply use 4-tuple, 5-tuple, etc. to avoid
confusion.

\begin{table}[H]
  \centering
  \begin{tabular}{ll}
  \toprule
  \textbf{Tuple Length} & \textbf{Names} \\
  \midrule
  0 & empty tuple \\
  1 & singleton, single \\
  2 & ordered pair, pair \\
  3 & triple, triplet \\
  4 & quadruple \\
  5 & quintuple \\
  6 & sextuple \\
  7 & septuple \\
  8 & octuple \\
  \( n  \) & n-tuple \\
  \bottomrule
  \end{tabular}
  \caption{Names for Common Tuples}
\end{table}

\begin{advancedTopic}
  The concept of tuples has been extended to infinite length. For more
  information, look up the topic of \emph{Hilbert spaces}.
\end{advancedTopic}

\begin{definition}{Tuple Equality}{tupleEquality}
  Two tuples \( A \) and \( B \) are equal if and only if they have the same length and
  their corresponding elements are equal.
\end{definition}

The following table contains examples of comparison of tuples for equality.

\begin{table}[H]
  \centering
  \begin{tabular}{ll}
    \toprule
    \textbf{Tuples} & \textbf{Equality Assertion} \\
    \midrule
    \( A = (1, 2, 3),\ B = (1, 2, 3) \) & \( A = B \) \\
    \( A = (1, 2, 3),\ B = (3, 2, 1) \) & \( A \neq B \) \\
    \( A = (x, y, z),\ B = (x, y, z) \) & \( A = B \) \\
    \( A = (\text{Apple}, \text{Orange}),\ B = (\text{Orange}, \text{Apple}) \) & \( A \neq B \) \\
    \( A = (1, 2, 3),\ B = (1, 2) \) & \( A \neq B \) \\
    \( A = (\{1, 2\}, \{3\}),\ B = (\{1, 2\}, \{3\}) \) & \( A = B \) \\
    \( A = (1, \emptyset, \{2\}),\ B = (1, \{\,\}, \{2, 3\}) \) & \( A \neq B \) \\
    \bottomrule
  \end{tabular}
  \caption{Tuple Equality Examples}
\end{table}
