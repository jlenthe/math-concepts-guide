\section{Relations}
Relations are used to describe how elements of various sets
are related to each other. For example, comparisons of numbers such as
\( a < b \) can be defined as relations.

\begin{definition}{Relation}{relation}
  A \textbf{relation} on sets \( A \) and \( B \) is any subset of the
  Cartesian product \( A \times B \).
\end{definition}

A relation gives a way to express how elements of one set are related to
each other (or not). The most commonly used relations are the standard
comparison operators such as \(<\), \(>\) and \(=\) for numbers. But
relations may also be defined for any kind of object. For example, in genealogy
we might define a set of people, \( \mathbb{P} \), representing some
population. Then we can define a relation for a concept such as
\textit{is ancestor of} and even define that formally using set builder
notation. We might introduce a symbol for this relation such as \( \prec \).
Then we can make expressions like
\[
  a \prec b \prec c
\]
\noindent
to express specific genealogical relationships unambiguously. Other
similar genealogical relations such as \textit{is a descendant of} or
\textit{is sibling of} can similarly be defined.

Many such real-world relationships among objects can be modeled using
mathematical relations. Some examples include:
\begin{itemize}
  \item Social networks: \textit{follows}, \textit{is friends with}
    \item Business and economic relationships: \textit{is supplier of},
      \textit{is customer of}, \textit{is subsidiary of}
    \item Species predation: \textit{preys on}, \textit{is preyed on by}
\end{itemize}

Relations often have special properties that are useful to identify. This
section will define the most commonly used of these.

\begin{definition}{Reflexive}{reflexive}
  A relation \( R \) on set \( A \) is \textbf{reflexive} if for all
  \( a \in A \), \( (a, a) \in R \).
\end{definition}

\begin{definition}{Irreflexive}{irreflexive}
  A relation \( R \) on set \( A \) is \textbf{irreflexive} if for all
  \( a \in A \), \( (a, a) \notin R \).
\end{definition}

A reflexive relation is one where every element is related to itself. An
irreflexive relation is one where no element is related to itself. A relation
that is neither reflexive nor irreflexive has some elements that are related
to themselves and some that are not.

\begin{definition}{Symmetric}{symmetric}
  A relation \( R \) on set \( A \) is \textbf{symmetric} if for all
  \( a, b \in A \), if \( (a, b) \in R \) then \( (b, a) \in R \).
\end{definition}

A symmetric relation is one where the relation has no sense of direction. It
indicates that two elements are intrinsically mutually related to each other
in some way.

\begin{definition}{Transitive}{transitive}
  A relation \( R \) on set \( A \) is \textbf{transitive} if for all
  \( a, b, c \in A \), if \( (a, b) \in R \) and \( (b, c) \in R \) then
  \( (a, c) \in R \).
\end{definition}

A transitive relation implies some kind of overarching structure or relatability
among all the elements such as the ability to strictly order numbers in a
consistent way.

\begin{definition}{Equivalence Relation}{equivalence-relation}
  A relation \( R \) is an \textbf{equivalence relation} if it is reflexive,
  symmetric, and transitive.
\end{definition}

An equivalence relation, as defined above, specifies the properties a relation
must have to embody the general concept of "equivalence", but it does not
specify the full meaning of a particular equivalence. Multiple equivalence relations
can be defined on the same set with different meanings. For example, on the set of
natural numbers \( \mathbb{N} \), we can define equivalence relations based on
numeric quantity, parity (even or odd), or divisibility. All such relations would
have the properties of reflexivity, symmetry, and transitivity.

For example, we could define an equivalence relation on \( \mathbb{N} \) that corresponds
to normal numerical equality, but we could also define parity equality as an equivalence
relation where two natural numbers are "equivalent" if they are both even or both odd.

\begin{table}[H]
  \centering
  \begin{tabular}{p{2in} p{3in}}
  \toprule
  \textbf{Comparison Relation} & \textbf{Properties} \\
  \midrule
  \( < \) (less than) & transitive, irreflexive \\
  \( \leq \) (less than or equal to) & reflexive, transitive \\
  \( > \) (greater than) & transitive, irreflexive \\
  \( \geq \) (greater than or equal to) & reflexive, transitive \\
  \( = \) (equal) & reflexive, symmetric, transitive \\
  \( \neq \) (not equal) & irreflexive, symmetric \\
  \bottomrule
  \end{tabular}
  \caption{Properties of Number Comparisons}
\end{table}
